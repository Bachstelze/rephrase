
\chapter{Conclusion}

\section{Summary}

During this project we developed and evaluated a paraphrasing tool that could be used as a part of a computer aided translation tool to assist translators. We started by analysing state of the art assistance techniques that are available in recent CAT implementations. Trying to relate these techniques to our approach we noticed that paraphrasing task could be efficiently solved using data produced by other assistance types. We continued our investigation by studying existing efforts in data-driven paraphrasing. We reviewed a paraphrasing approach that uses bilingual parallel corpora as data source \citep{Callison-Burch2007}. This approach has same high level idea as our solution. The idea is that if multiple items have same translation in foreign language, they are probably paraphrases. In contrast with previous studies, we exploited search graph data as our main source for paraphrasing. This data is being generated during decoding stage of the machine translation process.

We designed a simple paraphrasing service, which aims to find top paraphrases for user input. In Chapter 3 we described the preprocessing step which was carried out by us in order to boost graph querying performance. Next, we discussed importance of coverage information and how user input is being aligned with the foreign sentence. We reviewed the graph querying process, by introducing the coverage splitting problem and a solution we found to resolve it. Finally, our core idea was constructing a finite state machine using partial paraphrases retrieved from search graph and reduced using partial filters. This finite state machine is used to generate best paraphrasing options by using score functions to calculate transition scores between states. Finally, results are cleaned up using a set of filters and final ranking is produced by applying set of sorters. We implemented partial filters, score functions, filters and sorters as dynamically attachable functions. As a result our core implementation represents a framework that could easily be extended and modified. 

To test feasibility of our paraphraser we generated a large test case repository using machine translation corrupted sentences. In Chapter 4 we presented results of this automatic evaluation process for ten different versions of our paraphraser. We also commented on problems with this evaluation approach. Furthermore, we carried out a user study to confirm our previous results. The final outcome demonstrated that our best approach is significantly better than baseline approach which simply produces translation options without applying any sorters or filters.

Finally, in Chapter 5 we introduced additional features that aim to provide better paraphrasing service by taking advantage of the interactive environment. We described our implementation of a feature that allows to instantly update results based on a user feedback. Moreover, we discussed various situations in which paraphrasing tool might be useful and introduced a feature that lets users to request more specific paraphrasing.

In conclusion, we demonstrated how paraphrasing could be efficiently implemented within a computer assisted translation system, by reusing data generated during machine translation. 

\section{Future work}

One of the main outcomes of this work is the modular design of paraphraser, which allows to re-implement any part of the paraphrasing process. We previously demonstrated how new features like user feedback and specified requests handling could be easily supported by introducing new filters and sorters. Considering this we believe that our approach can be used as a core for more novel types of paraphrasing. Another direction for future work is collecting paraphrasing usage logs and investigating possibility of using them as training data to improve machine translation. Finally, it might be feasible to extend our approach to support data sources other than search graph, this way it will be possible to consider paraphrasing outside of the translation context.  