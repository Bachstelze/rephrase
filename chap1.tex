\chapter{Introduction}
  
\section{Motivation}

Today, taking advantage of achievements in machine translation, we can read and understand text in languages unknown to us. However, high quality translation still needs involvement of human translators. Computer Aided Translation (CAT) aims to support and assist translators, providing them with tools that partially automate and simplify their workflow. Demand for these tools is huge, each year international organisations such as European Union spend billions on translation services. 

Over past years researchers introduced several ways to aid translators. The most popular ones are interactive machine translation and post-editing. The first approach provides translation suggestions as user types, while the second approach initiates input area with a machine translation. 

During current project we developed and evaluated a novel assistance way that aims to help translators to improve a machine translation, by paraphrasing any user-specified part of the sentence.

Paraphrasing is a process of representing the same idea with different words. Some recent publications suggest that paraphrasing and translation tasks are closely related. Moreover, information used to perform one task, can be useful for another \cite{Callison-Burch2007}. 

In this report we will introduce a paraphrasing approach that efficiently reuses outcomes of the machine translation. Also, we will describe how paraphrasing performance can be improved within an interactive translation environment, by considering user feedback on the results. Finally, we will present a novel automatic evaluation approach that we used to test feasibility and usefulness of our paraphrasing tool. To confirm the automatic evaluation results we carried out several user studies. We will share outcomes of these experiments and discuss how they correlate with our previous findings. 

\section{Overview}

In this section we provide an outline of the report. We will start Chapter 2 by introducing background material that aims to provide required information about the context of our project. We will describe recent advancements in Computer Aided Translation and relate existing assistance tools to our paraphrasing service. Next we will provide references to previous publications on paraphrasing. We will also discuss some basic concepts of machine translation, focusing on the decoding process. The data structure generated during this process is known as search graph and it is used by us as the main source for paraphrasing. 

In Chapter 3 we will describe the work that was carried out in scope of current project. Firstly, we will start by discussing design of our approach and possible ways of integration with existing CAT tools. We will continue by providing implementation details for each component of our initial design. In this chapter we will also focus on challenging problems, that were discovered during design and implementation stages, and their solutions.

We will present our evaluation methodology in Chapter 4. We will provide description of an automatic evaluation process which was carried out by us during the project. Furthermore, we will presents evaluation results for ten different versions of our paraphrasing algorithm. We will continue by discussing advantages and disadvantages of the automatic evaluation. Finally, we will provide results of user studies that confirm our previous results.

In Chapter 5 we will present an improved paraphrasing approach that responds to user feedback in order to provide more relevant results. We will also describe a modification that we applied to our automatic evaluation tool in order to support this new feature. We will finish this chapter by introducing another interactive feature, that lets users to make specified paraphrasing requests.  

Finally, in Chapter 6 we will conclude our report and suggest interesting directions for future work. 
